%%%%%%%%%%%%%%%%%%%%%%%%%%%%%%%%%%%%%%%%%%%%%%%%%%%%%%%%%%%
% FILE: functions.tex
% DESC: Displays supported connection types between 
%       different Strimzi components
%
% TODO:
%   - Fix label sizing/orientation
%   - Refactor
%       * Organize functions into files based on topic
%   - Automatic Legend Layout
%       * Calc page length, node length, array length 
%%%%%%%%%%%%%%%%%%%%%%%%%%%%%%%%%%%%%%%%%%%%%%%%%%%%%%%%%%%
\usepackage[T1]{fontenc}
\usepackage[utf8]{inputenc}
\usepackage{lmodern}
\usepackage{tikz}
\usepackage{ifthen}
\usetikzlibrary{intersections}
\usetikzlibrary{positioning}
\usetikzlibrary{fit}
\usetikzlibrary{shapes}
\usetikzlibrary{calc}
\usetikzlibrary{matrix}
\usepackage{fontawesome} 
\usetikzlibrary{backgrounds}

\pgfdeclarelayer{background}
\pgfdeclarelayer{foreground}
\pgfsetlayers{background,main,foreground}
  
\definecolor{ClusterOperatorColor}{RGB}{0,127,189}  
\definecolor{TopicOperatorColor}{RGB}{0,51,71}
\definecolor{ZookeeperColor}{RGB}{11,102,35}
\definecolor{HardwareColor}{RGB}{0,0,51}

\tikzstyle{SECURE_CONN} =  [color=gray!60, thick, dashed, draw]
\tikzstyle{INSECURE_CONN} = [color=red!60, thick, dashdotted, draw] 

%\iffalse
%\fi

% Credit  https://tex.stackexchange.com/questions/37709/how-can-i-know-if-a-node-is-already-defined
\makeatletter
\long\def\ifnodedefined#1#2#3{%
    \@ifundefined{pgf@sh@ns@#1}{#3}{#2}%
}
\makeatother

% Establishes consistent variable scheme by prepending all
% variables with command name
\newcommand{\initVars}[4] {% {Prefix}{Id}{Size}{Location}
  \def\Prefix{#1}
  \expandafter\def\csname \Prefix Id\endcsname{#2}
  \expandafter\def\csname \Prefix Size\endcsname{#3}
  \tikzstyle{\Prefix Location} = [#4]
}

\newcommand{\kafka}[4] {% {Id}{Size}{Location}
  \initVars{kafka}{#1}{#2}{#3}
  
  \pgfmathsetmacro\msize{\kafkaSize*0.15}
  \pgfmathsetmacro\thickness{\msize*2.0}
  \pgfmathsetmacro\cthickness{\thickness*0.9} 
  \pgfmathsetmacro\dis{\msize * 0.2} 
   
  \tikzstyle{b} = [circle, inner sep=0pt, thick, line width=\thickness mm, minimum size=\msize cm]
  \tikzstyle{m} = [b, draw= black, fill = white]

  \node[ kafkaLocation,
         minimum width=\kafkaSize cm, 
         minimum height=\kafkaSize cm
       ] (\kafkaId) {};
  
  \begin{pgfonlayer}{foreground}
    \node[m, above = 0 cm of \kafkaId.center, anchor=center] (\kafkaId_M) {};
    \node[m, above = \dis cm of \kafkaId_M] (\kafkaId_m0) {};
    \node[m, above right = \dis cm of \kafkaId_M.15] (\kafkaId_m1) {};
    \node[m, below right = \dis cm of \kafkaId_M.345] (\kafkaId_m2) {};
    \node[m, below = \dis cm of \kafkaId_M] (\kafkaId_m3) {};
    \node[b, above left = \dis cm of \kafkaId_M.165] (\kafkaId_m4) {};

    \foreach \x/\y in {90/0,30/1, 330/2, 270/3} {
      \draw [line width=\cthickness mm] (\kafkaId_M.\x) edge (\kafkaId_m\y);
    }
  \end{pgfonlayer}
  %\node[fit=(\kafkaId_M)(\kafkaId_m0)(\kafkaId_m1)(\kafkaId_m2)(\kafkaId_m3)(\kafkaId_m4)] (\kafkaId) {};
}

\newcommand{\kafkaPod}[4]{% {Id}{Size}{Location}{label}
  \initVars{kafkaPod}{#1}{#2}{#3}
  \tikzstyle{kafkaPodColor} = [fill=yellow]     % #4 = color
    
  \pgfmathsetmacro\kafkaPodSize{\kafkaPodSize} 
   % Can't pass \kafkaPodLocation 
  \kafka{\kafkaPodId_0}{\kafkaPodSize}{#3}{#4}
  \pgfmathsetmacro\kafkaPodSep{- \kafkaPodSize * 2.5} 
  
  \begin{pgfonlayer}{main}
    \node[
           circle, 
           inner sep = \kafkaPodSep mm,
           minimum size = \kafkaPodSize cm,
           color=black, 
           kafkaPodColor,
           draw=white,
           fit=(\kafkaPodId_0),
           label={[label distance=0cm]0:{\fontsize{\fontSize pt}{\fontSep pt}\selectfont #4}}
         ] (\kafkaPodId) {};
  \end{pgfonlayer}
}

\newcommand{\kafkaConnect}[4] {% {Id}{Size}{Location}
  \initVars{kafka}{#1}{#2}{#3}
  
  \pgfmathsetmacro\msize{\kafkaSize*0.15}
  \pgfmathsetmacro\thickness{\msize*2.0}
  \pgfmathsetmacro\cthickness{\thickness*0.9} 
  \pgfmathsetmacro\dis{\msize * 0.2} 
  \pgfmathsetmacro\ddis{\msize * 0.5} 
   
  \tikzstyle{b} = [circle, inner sep=0pt, thick, line width=\thickness mm, minimum size=\msize cm]
  \tikzstyle{m} = [b, draw= black, fill = white]

  \node[ kafkaLocation,
         minimum width=\kafkaSize cm, 
         minimum height=\kafkaSize cm
       ] (\kafkaId) {};
  
  \begin{pgfonlayer}{foreground}
    \node[m, above = 0 cm of \kafkaId.center, anchor=center] (\kafkaId_mcore) {};
      %\node[right = \ddis cm of \kafkaId_mcore] (\kafkaId_mcore_east) {};
      \node[left  = \ddis cm of \kafkaId_mcore] (\kafkaId_mcore_west) {};
    \node[m, above = \dis cm of \kafkaId_mcore] (\kafkaId_m0) {};
      %\node[right = \ddis cm of \kafkaId_m0] (\kafkaId_m0_east) {};
      \node[left  = \ddis cm of \kafkaId_m0] (\kafkaId_m0_west) {};
    \node[m, above right = \dis cm of \kafkaId_mcore.15] (\kafkaId_m1) {};
      \node[right = \ddis cm of \kafkaId_m1] (\kafkaId_m1_east) {};
    \node[m, below right = \dis cm of \kafkaId_mcore.345] (\kafkaId_m2) {};
      \node[right = \ddis cm of \kafkaId_m2] (\kafkaId_m2_east) {};
    \node[m, below = \dis cm of \kafkaId_mcore] (\kafkaId_m3) {};
      %\node[right = \ddis cm of \kafkaId_m3] (\kafkaId_m3_east) {};
      \node[left = \ddis cm of \kafkaId_m3] (\kafkaId_m3_west) {};
    \node[b, above left = \dis cm of \kafkaId_mcore.165] (\kafkaId_m4) {};
    
    \foreach \x/\y in {90/0,30/1, 330/2, 270/3} {
      \draw [line width=\cthickness mm] (\kafkaId_mcore.\x) edge (\kafkaId_m\y);
    }
    
    \foreach \x/\y in {0/core, 90/0, 30/1, 330/2, 270/3} {
      \ifnodedefined{\kafkaId_m\y_east}{
        \draw [line width=\cthickness mm] (\kafkaId_m\y.east) edge (\kafkaId_m\y_east);
      }{}
      \ifnodedefined{\kafkaId_m\y_west}{
        \draw [line width=\cthickness mm] (\kafkaId_m\y.west) edge (\kafkaId_m\y_west);
      }{}
    }
  \end{pgfonlayer}
  %\node[fit=(\kafkaId_M)(\kafkaId_m0)(\kafkaId_m1)(\kafkaId_m2)(\kafkaId_m3)(\kafkaId_m4)] (\kafkaId) {};
}

\definecolor{kafkaConnectPod}{RGB}{219,112,147}
%\definecolor{mycolor}{RGB}{255,51,76}
\newcommand{\kafkaConnectPod}[4]{% {Id}{Size}{Location}{label}
  \initVars{kafkaConnectPod}{#1}{#2}{#3}
  \tikzstyle{kafkaConnectPodColor} = [fill=kafkaConnectPod] % #4 = color
    
  \pgfmathsetmacro\kafkaConnectPodSize{\kafkaConnectPodSize} 
   % Can't pass \kafkaPodLocation 
  \kafkaConnect{\kafkaConnectPodId_0}{\kafkaConnectPodSize}{#3}{#4}
  \pgfmathsetmacro\kafkaConnectPodSep{- \kafkaConnectPodSize * 2.5} 
  
  \begin{pgfonlayer}{main}
    \node[ circle, 
           inner sep = \kafkaConnectPodSep mm,
           minimum size = \kafkaConnectPodSize cm,
           color=black, 
           kafkaConnectPodColor,
           draw=white,
           fit=(\kafkaConnectPodId_0),
           label={[label distance=0cm]0:{\fontsize{\fontSize pt}{\fontSep pt}\selectfont #4}}
         ] (\kafkaConnectPodId) {};
  \end{pgfonlayer}
}

\newcommand{\pod}[4] {% {Id}{Size}{Location}{color}
  \initVars{pod}{#1}{#2}{#3}
  \tikzstyle{podColor} = [#4] % #4 = color
  
  \pgfmathsetmacro\ppodSize{\podSize*1.75}
  \pgfmathsetmacro\conSize{\ppodSize*0.25}

  \tikzstyle{base} = [rectangle, inner sep = 0 mm, rectangle, draw, thick]
  \tikzstyle{vari} = [minimum height = \conSize cm, minimum width = \conSize cm, podLocation, podColor]
  
  % randomly generate color for pod
  %\pgfmathparse{rnd}
  %\pgfmathtruncatemacro{\cc}{(\operatorId)*0.4}
  %\xdefinecolor{rColor}{rgb}{\cc, 0.7, \pgfmathresult}
    
  \node[base, vari] (\podId_center) {};
  
  \node[ circle, 
         thick, 
         inner sep = 0 mm, 
         minimum size= \ppodSize cm, 
         draw, 
         podColor, 
         fit=(\podId_center)
       ] (\podId) {};
  
}

\newcommand{\operator}[4] {% {Id}{Size}{Location}{color}
  \initVars{operator}{#1}{#2}{#3}
  \tikzstyle{operatorColor} = [#4] % #4 = color
  
  \pgfmathsetmacro\bodyLength{\operatorSize*1.20}
  \pgfmathsetmacro\bodyWidth{\operatorSize*0.25}
  \pgfmathsetmacro\limbLength{\bodyLength*0.70}
  \pgfmathsetmacro\limbWidth{\bodyWidth*0.55}
  \pgfmathsetmacro\armLength{\limbLength*1.75}
  \pgfmathsetmacro\operatorHeadDis{\operatorSize * 0.38}
  
  \tikzstyle{base} = [fill=black, inner sep=0pt,  minimum size = \operatorSize, operatorColor]
  
  \tikzstyle{head} = [base, circle, minimum size = \operatorSize cm]
  \tikzstyle{body} = [base, rectangle, minimum height = \bodyLength cm,  minimum width = \bodyWidth cm]
  \tikzstyle{limb} = [base, rectangle, minimum height = \limbLength cm,  minimum width = \limbWidth cm]
  \tikzstyle{arm}  = [limb, minimum height = \armLength cm]
  \tikzstyle{leg}  = [limb, rotate=90]
   
  \node[ operatorLocation,
         circle,
         minimum height = \operatorSize,
         minimum width =  \operatorSize
         ] (\operatorId) {};
  
  \node[head, above = \operatorHeadDis cm of \operatorId.center] (head_\operatorId) {};
    
  \node[body, below = 0 cm of head_\operatorId.south] (body_\operatorId) {};
  
  %\node[arm,  rotate = 120,  above right = 0 cm of body.east, anchor = east] (r_arm) {};
  %\node[arm,  rotate = -120, above left  = 0 cm of body.west, anchor = east]  (l_arm) {};
  
  \node[arm,  rotate = 90,  above = 0cm of body_\operatorId.center, anchor = center] (arm_\operatorId) {};
  
  \node[leg, rotate = 315, below right = 0 cm of body_\operatorId.south] (r_leg_\operatorId) {};
  \node[leg, rotate = 225, below left  = 0 cm of body_\operatorId.south]  (l_leg_\operatorId) {};
  
  %\node[ 
  %      fit=(head_\operatorId)(body_\operatorId)(arm_\operatorId)(r_leg_\operatorId)(l_leg_\operatorId)
  %     ] (\operatorId_center) {};
}


\newcommand{\clusterOperator}[4] {% {Id}{Size}{Location}{Label}
  \initVars{clusterOperator}{#1}{#2}{#3}
  
  \pgfmathsetmacro\coSize{\clusterOperatorSize * 0.20}
  \pgfmathsetmacro\objSize{\coSize * 0.5}
  \pgfmathsetmacro\dis{\coSize * 1.5}

  \pgfmathsetmacro\clusterOperatorSep{- \clusterOperatorSize * 1.75}
  
  \node[ circle, 
         inner sep = \clusterOperatorSep mm,
         minimum height = \clusterOperatorSize cm,
         minimum width = \clusterOperatorSize cm,
         draw=white, 
         fill=ClusterOperatorColor,
         clusterOperatorLocation,
         label={[label distance=0cm]0:{\fontsize{\fontSize pt}{\fontSep pt}\selectfont #4}}
       ] (\clusterOperatorId) {};
  
  \tikzstyle{\clusterOperatorId Position} = [above = 0 cm of \clusterOperatorId.center, anchor=center]
  
  \begin{pgfonlayer}{foreground}
    \operator{\clusterOperatorId Op}{\coSize}{\clusterOperatorId Position}{color=white}
    \disk{\clusterOperatorId_d0}{\objSize}{left = \dis cm of \clusterOperatorId Op.center}{color=white}
    \pod{\clusterOperatorId_p0}{\objSize}{right = \dis cm of \clusterOperatorId Op.center}{color=white}
    %\cube{\coID_cube}{\coSize}{black}{above = \dis cm of \coID.north}; % location
  \end{pgfonlayer}
}

\newcommand{\topicOperator}[4] {% {Id}{Size}{Location}{label}
  \initVars{topicOperator}{#1}{#2}{#3}
  %\tikzstyle{toColor} = [#4] % #4 = color
    
  \pgfmathsetmacro\toSize{\topicOperatorSize * 0.20}
  \pgfmathsetmacro\dis{\topicOperatorSize * 0.25 }
  \pgfmathsetmacro\scale{\toSize / 3}

  \tikzstyle{partition} = [rectangle, inner sep = 0]
  \tikzstyle{grid} = [scale=\scale, thin, draw=white]
  
   \node[ circle, 
          inner sep = 0 mm,
          minimum height = \topicOperatorSize cm,
          minimum width = \topicOperatorSize cm, 
          draw=white,
          fill=TopicOperatorColor,
          topicOperatorLocation,
          label={[label distance=0cm]0:{\fontsize{\fontSize pt}{\fontSep pt}\selectfont #4}}
        ] (\topicOperatorId) {};
  
  \tikzstyle{\topicOperatorId Position} = [above = 0 cm of \topicOperatorId.center, anchor=center]
  
  \operator{\topicOperatorId Op}{\toSize}{\topicOperatorId Position}{fill=white}
  
  \begin{pgfonlayer}{foreground}
    \node (rpart) [partition, right = \dis cm of \topicOperatorId Op.center]
    {
      \tikz{\draw[grid] (0,0)  grid (1,5);}
    };
    \node (lpart) [partition, left = \dis cm of \topicOperatorId Op.center] (lpart)
    {
      \tikz{\draw[grid] (0,0)  grid (1,5);}
    };
  \end{pgfonlayer}
  
}

\newcommand{\hardware}[3] {% {Id}{Size}{Location}
  \initVars{hardware}{#1}{#2}{#3}
  
  \pgfmathsetmacro\width{\hardwareSize*2.0}
  \pgfmathsetmacro\height{\hardwareSize*0.5}
  \pgfmathsetmacro\osHeight{\height*3}
  \pgfmathsetmacro\nodeDis{\hardwareSize*0.25}
  
  \tikzstyle{base} = [rectangle, rounded corners, minimum height = \height cm, minimum width = \width cm] 
  
  \node[hardwareLocation] (\hardwareId) {};
  %\node[base, fill=black!80, text=white, below = \nodeDis cm of \hardwareId] (\hardwareId_Node) {\tiny Node};
   \node[base, fill=black!80, text=white, below = \nodeDis cm of \hardwareId] (\hardwareId_Node) {Node};
   
  \begin{pgfonlayer}{background}
    \node [ base, 
            fill=gray!80, 
            text=white, 
            above = 0cm of \hardwareId_Node.north, 
            minimum height = \osHeight cm
          ] (\hardwareId_openshift) {};
  \end{pgfonlayer}

  %\node[above = 0cm of \hardwareId_openshift.south, text=white,
  %      label={[label distance=0cm]0:{\fontsize{\fontSize pt}{\fontSep pt}\selectfont Op}}
  %     ] () {OpenShift};  
  \node[above = 0cm of \hardwareId_openshift.south, text=white] () {OpenShift};
  %\node[above = 0cm of \hardwareId_openshift.south, text=white] () {\tiny OpenShift};
}

\newcommand{\zookeeper}[4] { %{Id}{Size}{Location}{label}
  \initVars{zookeeper}{#1}{#2}{#3}
  \pgfmathsetmacro\scale{\zookeeperSize}
  \pgfmathsetmacro\zookeeperFontSize{\zookeeperSize * 15}

  \node[ zookeeperLocation, 
         minimum width=\zookeeperSize cm, 
         minimum height=\zookeeperSize cm,
         fill=ZookeeperColor,
         circle,
         draw=white,
         text=white,
         font=\fontsize{\zookeeperFontSize}{0}\selectfont,
         label={[label distance=0cm]0:{\fontsize{\fontSize pt}{\fontSep pt}\selectfont #4}}
       ] (\zookeeperId) {Z};
}

\newcommand{\conn}[5] { %{Id}{Size}{Location}{Conn Type}{label}
   \initVars{conn}{#1}{#2}{#3}
   
   \node[ rectangle, 
          minimum height=\connSize cm, 
          minimum width=\connSize cm, 
          tlsLocation,
          label={[label distance=0cm]0:{\fontsize{\fontSize pt}{\fontSep pt}\selectfont #5}}
        ] (\connId) {};
   \path[-, #4] (\connId.west) edge (\connId.east);
}

\newcommand{\tls}[4] { %{Id}{Size}{Location}{label}
   \initVars{tls}{#1}{#2}{#3}
   \conn{#1}{#2}{#3}{S_CONN}{#4}
}

\newcommand{\nonTls}[4] { %{Id}{Size}{Location}{label}
   \initVars{tls}{#1}{#2}{#3}
   \conn{#1}{#2}{#3}{IS_CONN}{#4}
}

\newcommand{\configMap}[4] {% {Id}{Size}{Location}{Label}
  \initVars{configMap}{#1}{#2}{#3}
  
  \pgfmathsetmacro\scale{\configMapSize * 1.3}
  % TODO Find size mapping scheme of icons and remove magic number
  \pgfmathsetmacro\wrenchSize{2.5 + \configMapSize * 0.75}
  \definecolor{blueprint}{RGB}{4,63,140}

   \node[ configMapLocation,
          minimum width=\configMapSize cm, 
          minimum height=\configMapSize cm,
          rectangle,
          fill=blueprint,
          rounded corners,
          inner sep = 0 mm,
          draw=white,
          label={[label distance=0cm]0:{\fontsize{\fontSize pt}{\fontSep pt}\selectfont #4}}
        ] (\configMapId) {};
        
  %\draw (0,0) node (wrench) [yscale=\scale,xscale=\scale, thick, cmLocation ] {\bcoutil};
  \node[ yscale=\wrenchSize, 
         xscale=\wrenchSize,
         above = 0 cm of \configMapId.center,
         anchor = center, 
         color=white
      ] (wrench) {\faWrench};
}

\newcommand{\legend}[5] { %{Id}{Size}{Location}{Length}{LegendList}
  \initVars{legend}{#1}{#2}{#3}
  \pgfmathsetmacro\legendDis{\legendSize * 3.0}
  \pgfmathsetmacro\fontSize{\legendSize * 5}
  \pgfmathsetmacro\fontSep{\legendSize * 3}
  \pgfmathsetmacro\horzScale{\legendSize * 1.5}
  \pgfmathsetmacro\vertScale{\legendSize * 1.3}
  \pgfmathsetmacro\horzSep{#2 * 3.5}
                   
  \pgfmathsetmacro\horzDis{0}
  \pgfmathsetmacro\vertDis{0}
 
  \node[rectangle, minimum width = #4 cm, legendLocation] (LOC) {};

  \foreach\key/\id/\labe [count=\xi] in \legendList {
        
    \key{\id}
        {\legendSize}
        {above = 0cm of LOC.west, anchor=center, xshift = \horzDis cm, yshift = \vertDis cm}
        {\labe}
            
    \pgfmathtruncatemacro\horDis{\horzDis + \horzScale} 
    \ifthenelse { \horDis  < #4} 
    {
      \pgfmathparse{\horzDis + \horzSep}
      \global\let\horzDis=\pgfmathresult   
    }
    {
      \pgfmathparse{\horzDis - \horzDis}
      \global\let\horzDis=\pgfmathresult  
       
      \pgfmathparse{\vertDis - \vertScale}
      \global\let\vertDis=\pgfmathresult
    }
  }
   
  % Make List Here (Aggregate nodes for Legend )
  \node[dotted, thick, fit=(toL)(kpL)(zL)] () {};
}

\newcommand{\cluster}[6] { %{Id}{Size}{Location}{Num}{Type}{Extra}
  \initVars{cluster}{#1}{#2}{#3}
  
  \pgfmathsetmacro\numOfNodes{#4}
  \pgfmathsetmacro\nodeSize{#2}
  \pgfmathsetmacro\circWidth{#2*2}
  \pgfmathsetmacro\degreeSlice{360 / #4}
  \pgfmathsetmacro\angle{90}

  `#6'
  \node[circle, minimum width = \circWidth cm, clusterLocation] (\clusterId CENTER) {};
  
  \foreach \id in {0,...,#4} {
    `#5{\clusterId \id}{\nodeSize}{above = 0cm of \clusterId CENTER.\angle, anchor=center}{}'
    \pgfmathparse{\angle + \degreeSlice}
    \global\let\angle=\pgfmathresult
  }
}

\newcommand{\rings}[2] { %{Text}{numOfRings}
  \pgfmathsetmacro\connSize{\clusterSize}
  \pgfmathsetmacro\fontSize{\clusterSize * 5}
  \pgfmathsetmacro\fontSep{\clusterSize * 6}
  \pgfmathsetmacro\circSize{\clusterSize*2}
  \pgfmathsetmacro\incrSize{\clusterSize*0.4}
  
  \foreach \conn in {1,...,#2} {
    \node[ SECURE_CONN,
           circle, 
           line width= \connSize mm,
           minimum width= \circSize cm,   
           align=center, 
           font=\fontsize{\fontSize pt}{\fontSep pt}\selectfont,
           text=black,
           clusterLocation 
         ] () {#1};
    \pgfmathparse{\circSize - \incrSize}
    \global\let\circSize=\pgfmathresult 
  }
}

\newcommand{\disk}[4] {% {Id}{Size}{Location}{color}
  \initVars{disk}{#1}{#2}{#3}
  \tikzstyle{diskColor} = [#4] % #4 = color
  
  \pgfmathsetmacro\height{\diskSize*1.5}
  \pgfmathsetmacro\width{\diskSize*1.3}
  \pgfmathsetmacro\sep{\diskSize*1}
  \pgfmathsetmacro\shift{\diskSize*0.8}
  
  \tikzstyle{base} = [cylinder, draw, rotate = 90, thick, inner sep = \sep mm, shift={(\shift,0)}]
  \tikzstyle{vari} = [minimum height = \height cm, minimum width = \width cm, diskLocation, diskColor]
  
  \node[base, vari] (\diskId) {};
}

\newcommand{\fluentd}[4] { %{Id}{Size}{Location}{label}
  \initVars{fluentd}{#1}{#2}{#3}
  \pgfmathsetmacro\scale{\fluentdSize *0.04}
  \pgfmathsetmacro\fluentdFontSize{\fluentdSize * 15}

  \node[ fluentdLocation, 
         minimum width=\fluentdSize cm, 
         minimum height=\fluentdSize cm,
         fill=blue!0,
         circle,
         draw=white,
         text=white,
         font=\fontsize{\fluentdFontSize}{0}\selectfont,
         label={[label distance=0cm]0:{\fontsize{\fontSize pt}{\fontSep pt}\selectfont #4}}
       ] (\fluentdId) {\includegraphics[width=\scale\textwidth]{fluentd.png}}; %0.05
}

\newcommand{\elasticsearch}[4] { %{Id}{Size}{Location}{label}
  \initVars{elasticsearch}{#1}{#2}{#3}
  \pgfmathsetmacro\scale{\elasticsearchSize *0.04}
  \pgfmathsetmacro\elasticsearchFontSize{\elasticsearchSize * 15}

  \node[ elasticsearchLocation, 
         minimum width=\elasticsearchSize cm, 
         minimum height=\elasticsearchSize cm,
         fill=blue!0,
         circle,
         draw=white,
         text=white,
         font=\fontsize{\elasticsearchFontSize}{0}\selectfont,
         label={[label distance=0cm]0:{\fontsize{\fontSize pt}{\fontSep pt}\selectfont #4}}
       ] (\elasticsearchId) {\includegraphics[width=\scale\textwidth]{elasticsearch.png}}; %0.05
}

\newcommand{\loggingPod}[4] { %{Id}{Size}{Location}{label}
  \initVars{loggingPod}{#1}{#2}{#3}
  \pgfmathsetmacro\scale{\loggingPodSize * 0.04}
  \pgfmathsetmacro\loggingPodFontSize{\loggingPodSize * 15}

  \node[ loggingPodLocation, 
         minimum width=\loggingPodSize cm, 
         minimum height=\loggingPodSize cm,
         fill=orange!70,
         circle,
         draw=white,
         text=white,
         font=\fontsize{\loggingPodFontSize}{0}\selectfont,
         label={[label distance=0cm]0:{\fontsize{\fontSize pt}{\fontSep pt}\selectfont #4}}
       ] (\loggingPodId) {\faAlignLeft}; %0.05
}

\newcommand{\loggy}[5] {% {Id}{Size}{Location}{Mouth Angle}{label}
  \initVars{loggy}{#1}{#2}{#3}
  \pgfmathsetmacro\loggySize{#2/2}
  \pgfmathsetmacro\loggySiz{#2 * 0.10}
  \tikzstyle{loggyColor}=[fill=blue!10]
  
  \tikzstyle{all}=[line width=0.0 mm]
  \tikzstyle{body}= [all, rectangle, draw=blue!10, loggyColor, minimum size=#2 cm, rounded corners=0.25cm, inner sep=0mm]
  \tikzstyle{parts}=[all, fill=white, draw=black, inner sep=0mm, minimum size = \loggySiz cm]
  \tikzstyle{eye}=[ parts,circle]
  
  \node[#3] (\loggyId) {};
  \node[body, below=0cm of \loggyId.center, anchor=center] (\loggyId body) {};
  \node[above=0cm of \loggyId body.south, anchor=north] (\loggyId label) {#5};
  
  \node[circle, minimum size = \loggySize cm, above = 0 cm of \loggyId.center, anchor=center
       ] (\loggyId core) {};
       
  \node[ parts,
         semicircle,
         minimum size = \loggySiz cm,
         inner sep=0mm,
         rotate=#4,
         above = 0 cm of \loggyId.center,
         anchor=center
       ] (\loggyId mouth) {};

  \node[ eye, 
         below = 0 cm of \loggyId core.125, 
         anchor=center
       ] (\loggyId left_eye) {};
  \node[ eye,
         below = 0 cm of \loggyId core.55, 
         anchor=center
       ] (\loggyId right_eye) {};
}

\newcommand{\cube}[4] {% {Id}{Size}{Location}{color}
  \initVars{cube}{#1}{#2}{#3}
  \def\color{#4}
  
  \pgfmathsetmacro\cSize{\cubeSize*0.75}
    
  \def\x{\cSize * 0.4}
  \def\y{\cSize * 0.5}
  \def\z{\cSize * 1.0}

  \tikzstyle{coor} = [shape=coordinate]
  \node[circle, cubeLocation] (\cubeId) {};

  \node[coor, below left =\x cm and \y cm of \cubeId] (S_\cubeId) {S};
  \node[coor, above=\z cm and 0cm of S_\cubeId] (A_\cubeId) {A};
  \node[coor, above right=\y cm and \x cm of A_\cubeId] (B_\cubeId) {B};
  \node[coor, above right=\y cm and \x cm of S_\cubeId] (C_\cubeId) {C};
  \node[coor, right=0.0cm and \z cm of S_\cubeId] (D_\cubeId) {D};
  \node[coor, above=\z cm and 0.0cm of D_\cubeId] (E_\cubeId) {E};
  \node[coor, above right=\y cm and \x cm of E_\cubeId] (F_\cubeId) {F};
  \node[coor, above right=\y cm and \x cm of D_\cubeId] (G_\cubeId) {G};

  \tikzstyle{face} = [black]

  \draw[face, fill=\color!30] (S_\cubeId) -- (C_\cubeId) -- (G_\cubeId) -- (D_\cubeId) -- cycle; % Bottom Face
  \draw[face, fill=\color!30] (S_\cubeId) -- (A_\cubeId) -- (E_\cubeId) -- (D_\cubeId) -- cycle; % Back Face
  \draw[face, fill=\color!10] (S_\cubeId) -- (A_\cubeId) -- (B_\cubeId) -- (C_\cubeId) -- cycle; % Left Face
  \draw[face, fill=\color!20,opacity=0.8] (D_\cubeId) -- (E_\cubeId) -- (F_\cubeId) -- (G_\cubeId) -- cycle; % Right Face
  \draw[face, fill=\color!20,opacity=0.6] (C_\cubeId) -- (B_\cubeId) -- (F_\cubeId) -- (G_\cubeId) -- cycle; % Front Face
  \draw[face, fill=\color!20,opacity=0.8] (A_\cubeId) -- (B_\cubeId) -- (F_\cubeId) -- (E_\cubeId) -- cycle; % Top Face

  \node[coor] (\cubeId_center) at ($(S_\cubeId)!0.5!(F_\cubeId)$)  {};
  \node[coor] (\cubeId_north)  at ($(A_\cubeId)!0.5!(F_\cubeId)$)  {};
  \node[coor] (\cubeId_east)   at ($(D_\cubeId)!0.5!(F_\cubeId)$)  {};
  \node[coor] (\cubeId_south)  at ($(S_\cubeId)!0.5!(G_\cubeId)$)  {};
  \node[coor] (\cubeId_west)   at ($(S_\cubeId)!0.5!(B_\cubeId)$)  {};

  %\node[fit=(S_\cubeId)(F_\cubeId)] (\cubeId) {};
  %\tikzstyle{cube}= [fit=(S_\cubeId)(A_\cubeId)(B_\cubeId)(C_\cubeId)(D_\cubeId)(E_\cubeId)(F_\cubeId)(G_\cubeId)];
  %\node[circle, draw, cubeLocation] (test\cubeId) {};
}
